\documentclass{article}
\usepackage[utf8]{inputenc}
\usepackage{graphicx}
\usepackage[colorlinks=true, allcolors=black]{hyperref}
\usepackage{listings}
\usepackage[a4paper, total={6in, 8in}]{geometry}
\usepackage{pdfpages}
\usepackage{float}
\usepackage{soul}
\usepackage{multicol}
\usepackage{multirow}
\usepackage[section]{placeins}
\usepackage[
backend=biber,
sorting=none]{biblatex}

\graphicspath{ {./img/} }

\addbibresource{bibliography.bib}

\NewDocumentCommand{\codeword}{v}{%
\texttt{\textcolor{black}{#1}}%
}

\title{CS5011: P2 - Machine Learning}
\author{190018035}
\date{March 13th 2024}

% PAGE LIMIT: 10, including text, figures, tables

\begin{document}

\maketitle

\tableofcontents

\section{Introduction}
% Include checklist of parts completed & extension

\subsection{Project Achievements}
\begin{itemize}
    \item{Part 1: Attempted and Fully Working}
    \item{Part 2: Attempted and Fully Working}
\end{itemize}

\subsection{Usage Instructions}

\section{Design and Implementation}
% Explain the various preprocessing steps you have done.
% Which design decisions (parameters of preprocessing methods, hyper-parameters of ma-
% chine learning models) you have to make while implementing this part? How did you
% set them? To make it simple, you can start your analysis with the default setting, and
% investigate a few hyper-parameters of one or two machine learning models of your choice
% afterwards.

\section{Evaluation}
% Analyse the results of your experiments based on the provided training set (via cross-
% validation mechanism). Summarise the key insights and findings of your analysis.

\section{Conclusion}

\printbibliography

\end{document}
